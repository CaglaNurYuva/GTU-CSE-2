\documentclass[a4 paper]{article}
\usepackage[inner=2.0cm,outer=2.0cm,top=2.5cm,bottom=2.5cm]{geometry}
\usepackage{setspace}
\usepackage[ruled]{algorithm2e}
\usepackage[rgb]{xcolor}
\usepackage{verbatim}
\usepackage{subcaption}
\usepackage{amsgen,amsmath,amstext,amsbsy,amsopn,tikz,amssymb,tkz-linknodes}
\usepackage{fancyhdr}
\usepackage[colorlinks=true, urlcolor=blue,  linkcolor=blue, citecolor=blue]{hyperref}
\usepackage[colorinlistoftodos]{todonotes}
\usepackage{rotating}
\usepackage{booktabs}
\newcommand{\ra}[1]{\renewcommand{\arraystretch}{#1}}

\newtheorem{thm}{Theorem}[section]
\newtheorem{prop}[thm]{Proposition}
\newtheorem{lem}[thm]{Lemma}
\newtheorem{cor}[thm]{Corollary}
\newtheorem{defn}[thm]{Definition}
\newtheorem{rem}[thm]{Remark}
\numberwithin{equation}{section}

\newcommand{\homework}[6]{
   \pagestyle{myheadings}
   \thispagestyle{plain}
   \newpage
   \setcounter{page}{1}
   \noindent
   \begin{center}
   \framebox{
      \vbox{\vspace{2mm}
    \hbox to 6.28in { {\bf CSE 211:~Discrete Mathematics \hfill {\small (#2)}} }
       \vspace{6mm}
       \hbox to 6.28in { {\Large \hfill #1  \hfill} }
       \vspace{6mm}
       \hbox to 6.28in { {\it Instructor: {\rm #3} \hfill Name: {\rm #5}Çağla Nur Yuva \hfill Student Id: {\rm #6}}200104004081 \hfill}
       \hbox to 6.28in { {\it Assistant: #4  \hfill #6}}
      \vspace{2mm}}
   }
   \end{center}
   \markboth{#5 -- #1}{#5 -- #1}
   \vspace*{4mm}
}

\newcommand{\problem}[2]{~\\\fbox{\textbf{Problem #1}}\hfill (#2 points)\newline\newline}
\newcommand{\subproblem}[1]{~\newline\textbf{(#1)}}
\newcommand{\D}{\mathcal{D}}
\newcommand{\Hy}{\mathcal{H}}
\newcommand{\VS}{\textrm{VS}}
\newcommand{\solution}{~\newline\textbf{\textit{(Solution)}} }

\newcommand{\bbF}{\mathbb{F}}
\newcommand{\bbX}{\mathbb{X}}
\newcommand{\bI}{\mathbf{I}}
\newcommand{\bX}{\mathbf{X}}
\newcommand{\bY}{\mathbf{Y}}
\newcommand{\bepsilon}{\boldsymbol{\epsilon}}
\newcommand{\balpha}{\boldsymbol{\alpha}}
\newcommand{\bbeta}{\boldsymbol{\beta}}
\newcommand{\0}{\mathbf{0}}


\begin{document}
\homework{Homework \#1}{Due: 30/10/22}{Dr. Zafeirakis Zafeirakopoulos}{Başak Karakaş}{}{}
\textbf{Course Policy}: Read all the instructions below carefully before you start working on the assignment, and before you make a submission.
\begin{itemize}
\item It is not a group homework. Do not share your answers to anyone in any circumstance. Any cheating means at least -100 for both sides. 
\item Do not take any information from Internet.
\item No late homework will be accepted. 
\item For any questions about the homework, send an email to bkarakas2018@gtu.edu.tr
\item Use LaTeX. You can work on the tex file shared with you in the assignment document.
\item Submit both the tex and pdf files into Homework1. Name of the files should be "\emph{SurnameName$\_$Id.tex}" and "\emph{SurnameName$\_$Id.pdf}".
\end{itemize}

\problem{1: Sets}{3+3+3+3+3=15}
Which of the following sets are equal? Show your work step by step.\newline
\subproblem{a} $\{$t : t is a root of $x^2$ – 6x + 8 = 0$\}$
\newline
\subproblem{b} $\{$y : y is a real number in the closed interval [2, 3]$\}$
\newline
\subproblem{c} $\{$4, 2, 5, 4$\}$
\newline
\subproblem{d} $\{$4, 5, 7, 2$\}$ - $\{$5, 7$\}$
\newline
\subproblem{e} $\{$q: q is either the number of sides of a rectangle or the number of digits in any integer between 11 and 99$\}$\\
\solution
\newline
\\ 5 sets are going to be named as A,B,C,D and E respectively for the sake of clarity. 
\begin{itemize}
    \item (a) Let us find roots of the equation to find elements of the set.
Roots of the equation y=\(x^2-6x+8\) are 4 and 2. Therefore, A = \{4,2\}
    \item (b) B = \{2, ..., 3\} - $\mathbb{C}$
    \item (c) C = \{4,2,5,4\}\\
    \item (d) D = \{4,2\}  5 and 7 are not included.\\
    \item (e) E = \{4,2\} \\
\end{itemize}

Set A, set D and set E are considered as equal set of elements since they contain the same elements. Therefore;\\
\[A = D = E\]



\newpage
\problem{2: Cardinality of Sets}{2+2+2+2=8}
What is the cardinality of each of these sets? Explain your answers.\\
\subproblem{a} $\{\emptyset\}$\\
\subproblem{b} $\{\emptyset,\{\emptyset\}\}$\\
\subproblem{c} $\{\emptyset,\{\emptyset,\{\emptyset\}\}\}$\\
\subproblem{d} $\{\emptyset,\{\emptyset,\{\emptyset,\{\emptyset\}\}\}\}$\\
\solution
\newline
\newline
4 sets are going to be named as A,B,C,D respectively for the sake of clarity. Cardinality of a set is the number of objects in the set. This object can be a set, an individual element, set of subsets and even an empty set. 
\begin{itemize}
    \item (a) The only element in set A is an empty set, therefore $|A|=1$
    \item (b) There are two elements in set B, first one is an empty set, second one is a set holding an empty set, therefore $|B|=2$
    \item (c) There are two elements in set C, first one is an empty set, second one is a set containing two elements, therefore $|C|=2$
    \item (d) There are two elements in set D, first one is an empty set, second one is a set containing two elements, therefore $|D|=2$
\end{itemize}


\problem{3: Cartesian Product of Sets}{15}
Explain why (A $\times$ B) $\times$ (C $\times$ D) and A $\times$ (B $\times$ C) $\times$ D are not the same.\\
\solution
\newline
\\Let us assume that;
\begin{itemize}
    \item A = \{a: a $\in$ A\}
    \item B = \{b: b $\in$ B\}
    \item C = \{c: c $\in$ C\}
    \item D = \{d: d $\in$ D\}
\end{itemize}

There are set A, set B, set C and set D, which all are non-empty. Therefore,
\begin{itemize}
    \item A x B = \{(a,b): a $\in$ A, b $\in$ B\}
    \item C x D = \{(c,d): c $\in$ C, d $\in$ D\}\\ 
\end{itemize}

Therefore,
\begin{itemize}
    \item (A x B) x (C x D) = \{(a,b), (c,d): (a,b) $\in$ (A x B), (c,d) $\in$ (C x D)\}
\end{itemize}

Additionally,
\begin{itemize}
    \item B x C = \{(b,c): b $\in$ B, c $\in$ C\}
    \item A x (B x C) = \{(a, (b,c)): a $\in$ A, (b,c) $\in$ (B x C)\}
\end{itemize}

Therefore,
\begin{itemize}
    \item A x (B x C) x D = \{(a,(b,c),d): a $\in$ A, (b,c) $\in$ (B x C), d $\in$ D\}
\end{itemize}

As a result,
\begin{itemize}
    \item ((a,b),(c,d)) $\neq$ (a,(b,c),d)
    \item (A x B) x (C x D) $\neq$ A x (B x C) x D
\end{itemize}


\newpage
\problem{4: Cartesian Product of Sets in Algorithms }{25}
Let A, B and C be sets which have different cardinalities. Let (p, q, r) be each triple of A $\times$ B $\times$ C where p $\in$ A, q $\in$ B and r $\in$ C. Design an algorithm which finds all the triples that are satisfying the criteria: p $\leq$ q and q $\geq$ r. Write the pseudo code of the algorithm in your solution.\newline
\newline
For example: Let the set A, B and C be as A = $\{$ 3, 5, 7 $\}$, B = $\{$ 3, 6 $\}$ and C = $\{$ 4, 6, 9 $\}$. Then the output should be : $\{$ (3, 6, 4), (3, 6, 6), (5, 6, 4), (5, 6, 6) $\}$. \newline
\newline
(Note: Assume that you have sets of A, B, C as an input argument.)\newline
\solution

\begin{algorithm}
\SetAlgoLined
\KwIn{The sets of A, B, C}

\For{keep going through the elements of set A until all elements in set A has been gone through once}{
    \For{going through the elements of set B until all elements in set B has been gone through once}{
        \For{going through the elements of set C until all elements in set C has been gone through once}{
            \eIf{element of set A is less than or equal to element of set B and element of set B is greater than or equal to element of set C is true}{
                print particular element in set A, set B and set C respectively and put parentheses and commas as indicated in the expected output.
            }{
                keep going through the sets' elements
            }
        }
    }
}


\caption{Pseudo Code of Your Algorithm}
\end{algorithm}

\newpage
\problem{5: Functions}{16}
If f and f $\circ$ g are one-to-one, does it follow that g is one-to-one? Justify your answer.\\
\solution
\newline
\newline
Let us assume $g(a) = g(b)$. Unless a is equivalent to b, g cannot be a one-to-one function. Therefore, we will show a is equivalent to b or not.

\begin{itemize}
    \item Let us take the function f of each side of the previous equation:
\[f(g(a)) = f(g(b))\]   
    \item Using the definition of composition:
    \begin{center}
         f $\circ$ g(a) = f $\circ$ g(b) 
    \end{center}
    \item Since it is already said that f $\circ$ g is one-to-one, a is equivalent to b. By the definition of one-to-one function, we have proved that g is a one-to-one function.
     
\end{itemize}


\problem{6: Functions}{7+7+7=21}
Determine whether the function $f:$ $\mathbb{Z}\times\mathbb{Z}\to\mathbb{Z}$ is onto if\\
\subproblem{a} $f(m,n)=2m-n$\\
\subproblem{b} $f(m,n)=m^2-n^2$\\
\subproblem{c} $f(m,n)=\mid m\mid - \mid n\mid$\\
\solution
\newline
\begin{itemize}
    \item (a) Given any integer n, we have f(0,n) = -n, therefore the function is onto.\\
    \item (b) Given any integer n, we have f(0,n) = $-n^2$.  $n^2$  $\geq$ 0 and $-n^2$ $\leq$ 0. As seen, the range does not contain any positive integers, therefore the function is not onto.\\
    \item (c) Given any integer m, we have f(m,0) = $|m|$. $|m|$ $\geq$ 0. As seen, the range does not contain any negative integers. Therefore, the function is not onto.  
\end{itemize}

\newpage
\problem{7: Functions}{Bonus 20}
Suppose that $f$ is a function from $A$ to $B$, where $A$ and $B$ are finite sets with $\mid A\mid=\mid B\mid$. Show that $f$ is one-to-one if and only if it is onto.\\
\solution
\newline
\begin{itemize}
    \item Since f is one-to-one, every element in A is mapped to a distinct element in B, which means if f(x) = f(y), x=y for x,y $\in$ A.
    \item We will use method of contradiction to prove it. Let us say f is not onto, which means there is at least one element in set B with no preimage in set A. Therefore, $|B|$ will be at least one greater than $|A|$, which contradicts $|A|$ = $|B|$. Therefore, f is onto.
\end{itemize}


\end{document} 

